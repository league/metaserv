\documentclass{acm_proc_article-sp}
\usepackage{ucs,textcomp,url}
\usepackage[numbers]{natbib}
\usepackage[utf8]{inputenc}
\ifx\pdfoutput\undefined\else
  \usepackage[
    backref,
    pdfpagemode=None,
    pdftitle={MetaOCaml Server Pages},
    pdfsubject={Web Publishing as Staged Computation},
    pdfauthor={Christopher League}
  ]{hyperref}
\fi
\makeatletter
\usepackage{struktur,bodoni}

\iffalse
\renewcommand{\Large}{\sffamily\fontsize\@xivpt{18}}
\renewcommand\section{\@startsection{section}{1}{\z@}%
                       {-18\p@ \@plus -4\p@ \@minus -4\p@}%
                       {12\p@ \@plus 4\p@ \@minus 4\p@}%
                       {\sffamily\large\bfseries\boldmath
                        \rightskip=\z@ \@plus 8em\pretolerance=10000 }}
\renewcommand\subsection{\@startsection{subsection}{2}{\z@}%
                       {-18\p@ \@plus -4\p@ \@minus -4\p@}%
                       {8\p@ \@plus 4\p@ \@minus 4\p@}%
                       {\sffamily\normalsize\bfseries\boldmath
                        \rightskip=\z@ \@plus 8em\pretolerance=10000 }}
\fi

%% Following is for fiddling with acmproc-sp fonts
\font\ixpt      = 5bdrx7t at 9pt
\font\confname  = 5bdrix7t at 8pt
\font\crnotice  = 5bdrx7t at 8pt
\font\ninept    = 5bdrx7t at 9pt
\font\secfnt    = 5sxb7t at 12pt
\font\subsecfnt = 5sxb7t at 11pt
\font\ttlfnt    = 5sxh7t at 18pt
\font\subttlfnt = 5sxb7t at 14pt
\font\aufnt     = 5sxb7t at 11pt
\font\affaddr   = 5sxr7t at 10pt
\font\eaddfnt   = 5sxr7t at 10pt
\makeatother

%%% Local Variables: 
%%% mode: latex
%%% TeX-master: "paper"
%%% End: 

\begin{document}
\title{MetaOCaml Server Pages:}
\subtitle{Web Publishing as Staged Computation}
\numberofauthors{1}
\author{
  \alignauthor Christopher League\\
  \affaddr{Long Island University · Computer Science}\\
  \affaddr{1 University Plaza · Brooklyn, NY 11201}\\
  \email{christopher.league@liu.edu}}
\conferenceinfo{MetaOCaml Workshop}{October 2004, Vancouver}
\CopyrightYear{2004}
\maketitle
\begin{abstract}
...
\end{abstract}
\category{D.3.2}{Programming Languages}{Language
  Classifications}[Specialized application languages]
\category{H.3.5}{Information Storage and Retrieval}{Online Information
  Services}[Web-based services]
\section{Introduction}

Today, most successful sites are programs, not static
data~\cite{greenspun99panda}.

Web programming is difficult.

It is an application area that is naturally staged.  First, content is
prepared and placed on the server (publishing stage).  Next, a user's
browser requests the content (retrieval stage).  Finally, the browser
displays the content on the client (display stage).  There is an
opportunity for dynamic code to be executed at any of these stages.

Give an example: conference calendar.  

\section{Performance}

\section{Related work}

\subsection{Server-side computation}
\label{sec:related-server}

\begin{itemize}
\item PHP~\cite{bakken04php} is a neat language.
\item JSP~\cite{mahmoud03jsp}
\item SMLserver~\cite{elsman02smlserver,elsman03web}
\item XCaml~\cite{baretta04xcaml}
\end{itemize}

See also the book by \citet{greenspun99panda}.

\subsection{Client-side computation}
\label{sec:related-client}
\begin{itemize}
\item JavaScript \cite{ecmascript99,flanagan01javascript}
\item MMM, with Caml applets, by \citet{rouaix96web}
\end{itemize}

\subsection{Validated HTML in functional language}
\begin{itemize}
\item \citet{elsman04typing}
\item \citet{wallace99haxml}
\item \citet{ohl04xhtml}
\item \citet{hosoya03xduce}
\end{itemize}

\section{Future directions}

\section{Conclusion}

\bibliographystyle{abbrvnat}
\bibliography{refs,../../cal}

\end{document}

%%% Local Variables: 
%%% mode: latex
%%% TeX-master: t
%%% End: 
