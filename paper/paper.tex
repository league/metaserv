\documentclass{elsart}
\usepackage{textcomp,url,xspace,graphicx,color}
\def\UrlFont{\tt\small}
%\usepackage[utf8]{inputenc}
\usepackage[leftno,norules,nolineno]{lgrind}
%\usepackage[bf]{caption2}
\usepackage[numbers]{natbib}
\listfiles
\ifx\pdfoutput\undefined\else
  \usepackage[
    pdfpagemode=None,
    pdftitle={MetaOCaml Server Pages},
    pdfsubject={Web Publishing as Staged Computation},
    pdfauthor={Christopher League}
  ]{hyperref}
\fi
\makeatletter
\usepackage{struktur,bodoni}
\def\labelitemi{\textbullet}
\def\la{{\rawrm\char'074}}
\def\ra{{\rawrm\char'076}}
\iffalse
\renewcommand{\Large}{\sffamily\fontsize\@xivpt{18}}
\renewcommand\section{\@startsection{section}{1}{\z@}%
                       {-18\p@ \@plus -4\p@ \@minus -4\p@}%
                       {12\p@ \@plus 4\p@ \@minus 4\p@}%
                       {\sffamily\large\bfseries\boldmath
                        \rightskip=\z@ \@plus 8em\pretolerance=10000 }}
\renewcommand\subsection{\@startsection{subsection}{2}{\z@}%
                       {-18\p@ \@plus -4\p@ \@minus -4\p@}%
                       {8\p@ \@plus 4\p@ \@minus 4\p@}%
                       {\sffamily\normalsize\bfseries\boldmath
                        \rightskip=\z@ \@plus 8em\pretolerance=10000 }}
\fi

\def\opencode{{\rawhv\char'074}{\rawbf?}}
\def\closecode{{\rawbf?}{\rawhv\char'076}}
\lstdefinelanguage{meta}
  {morekeywords={and,as,begin,do,done,downto,else,end,exception,for,%
      fun,if,in,let,match,val,mutable,not,of,open,or,prefix,rec,then,%
      try,while,with},%
   morecomment=[n]{(*}{*)},%
   morestring=[b]",%
   escapeinside={(*@}{@*)},%
   literate=
     {->}{$\rightarrow\;$}2
     {<}{\la}1
     {>}{\ra}1
     {\\}{{\rawrm\char'134}}1
     {|}{{\rawrm\char'174}\thinspace}1
     {<=}{{\rawrm\char'235}\thinspace}1
     {"}{{\rawrm\char'042}}1
     {'}{{\rawrm\char'264}}1
     {<?}{{\color{green}\opencode}\thinspace}2
     {<?pragma}{{\color{green}\opencode}pragma\ }8
     {<?^}{{\color{green}\opencode{\rawhv\char'136}}}3
     {<?"}{{\color{green}\opencode{\rawrm\char'042}}}3
     {<?~}{{\color{green}\opencode{\rawhv\char'176}}}3
     {<?=}{{\color{green}\opencode{\rawhv\char'075}}\thinspace}3
     {<?~=}{{\color{green}\opencode{\rawhv\char'176\char'075}}\thinspace}4
     {?>}{\thinspace{\color{green}\closecode}}2
     {.<}{{\color{blue}\rawhv.\char'074}\thinspace}2
     {>.}{\thinspace{\color{blue}\rawhv\char'076.}}2
     {.~}{{\color{blue}\rawhv.\char'176}}2
   }

\definecolor{green}{rgb}{0,.6,0}
\definecolor{blue}{rgb}{0,0,.6}
\lstset{columns=fullflexible,
  basicstyle=\sffamily,
  keywordstyle=\bfseries, belowskip=0pt,
  numbers=left, numberstyle=\tiny, numbersep=4pt
}

\lstset{language=meta}

%% Following is for fiddling with acmproc-sp fonts
\font\rawrm     = 5sxr8r at 9pt
\font\rawbf     = 5sxb8r at 9pt
\font\rawhv     = 5sxh8r at 9pt
\font\ixpt      = 5bdrx7t at 9pt
\font\confname  = 5bdrix7t at 8pt
\font\crnotice  = 5bdrx7t at 8pt
\font\ninept    = 5bdrx7t at 9pt
\font\secfnt    = 5sxb7t at 12pt
\font\subsecfnt = 5sxb7t at 11pt
\font\ttlfnt    = 5sxh7t at 18pt
\font\subttlfnt = 5sxb7t at 14pt
\font\aufnt     = 5sxb7t at 11pt
\font\affaddr   = 5sxr7t at 10pt
\font\eaddfnt   = 5sxr7t at 10pt
\makeatother

%%% Local Variables: 
%%% mode: latex
%%% TeX-master: "paper"
%%% End: 

\def\MOC{MetaOCaml\xspace}
\hyphenation{Meta-OCaml data-base Slash-dot}
\def\quote{\list{}{\leftmargin1.4em\rightmargin1em}\item[]}
\newcommand{\myfig}[3]{%
  \begin{figure}[tbp]%
    #3%
    \caption{#2}%
    \label{fig:#1}%
  \end{figure}}
\newcommand{\screenshot}[2]{%
  \myfig{#1}{#2}
  {\centering%
    \includegraphics[width=25pc,keepaspectratio]{#1}}}
\newcommand{\gnuplot}[2]{%
  \myfig{#1}{#2}{\centering\input{#1}}}

\LGindent=16pt
\def\lgrindfile#1{
  \par\vskip.5\baselineskip\begingroup
  \begin{lgrind}\input #1\relax
  \end{lgrind}\endgroup}

\newcommand{\mylstf}[2]{%
  \myfig{#1}{\texttt{#1}.  #2}{\LGnuminterval=3\lgrindfile{#1}}}
\newcommand{\mylst}[2]{%
  \myfig{#1}{#2}{\LGnuminterval=3\lgrindfile{#1}}}

\begin{document}
\begin{frontmatter}
\title{\MOC Server Pages:\\Web Publishing as Staged Computation}
\author{Christopher League}
\address{Long Island University \textperiodcentered{} Computer Science\\
  1 University Plaza \textperiodcentered{} Brooklyn, NY 11201\\
  \textup{\texttt{christopher.league@liu.edu}}}

\begin{abstract}
  Modern dynamic web services are really computer programs.  Some
  parts of these programs run off-line, others run server-side on each
  request, and still others run within the browser.  In other words,
  web publishing is \emph{staged computation,} either for better
  performance, or because certain resources are available in one stage
  but not another.  Unfortunately, the various web programming
  languages make it difficult to spread computation over more than one
  stage.  This is a tremendous opportunity for multi-stage languages
  in general, and for \MOC in particular.
  
  We present the design of \MOC Server Pages.  Unlike other languages
  in its genre, the embedded \MOC code blocks may be annotated with
  staging information, so that the programmer may safely and precisely
  control which computation occurs in which stage.  A prototype web
  server, written in OCaml, supports web sites with both static and
  dynamic content.  We provide several sample programs and demonstrate
  the performance gains won using multi-stage programming.
\end{abstract}
\end{frontmatter}

%\category{D.3.2}{Programming Languages}{Language
%  Classifications}[Specialized application languages]
%\category{H.3.5}{Information Storage and Retrieval}{Online Information
%  Services}[Web-based services]

\section{Motivation}
\label{sec:intro}
Modern dynamic web sites support many features for user collaboration
and personalization.  To provide such services, web sites contain
custom computer programs, often written in one of a family of
programming languages that have grown up around (or been adapted for)
the web.

There is at least one dictum of program design that we cannot escape
on the web: \emph{performance matters.}  As a web publisher, visitors
are your livelihood.  But will your servers and scripts be ready for
the day that your site is featured on prime time television, or on
\texttt{slashdot.org}?  If tens of thousands of potential users drop
by to find a sluggish (or dead) server, most of them will never
return~\cite{greenspun99panda}.

This happens so often to sites featured on Slashdot---a
``news for nerds'' discussion site---that it has come to be
known as the \emph{Slashdot effect:} ``a site that might be
designed to handle a few hundred hits per day can suddenly
find itself handling that many a
second.''~\cite{turner03slashdot}~ Although some ISPs have
bandwidth limitations, Slashdot creator Rob Malda says most
sites that fail suffer from poor planning and architecture:
\begin{quote}
  Anybody who has a pretty good
  understanding of web design [has] done a good job of
  learning what information to cache [and] what needs
  to be pre-generated.  So when you're actually loading a
  page, even if it's a complicated page that looks dynamic
  and custom, on the back end of that, what they're really
  doing is putting together a bunch of puzzle pieces that
  have been pre-generated, and making the simplest, quickest
  decisions they possibly can.~\cite{turner03slashdot}
\end{quote}

Malda's observation points (informally) to the idea of \emph{staging}
the computation performed by a web service.  Indeed, web publishing is
an application area that is naturally staged:\footnote{The literature
on meta-programming has yet to acknowledge
  web services~\cite{w3c-wsdl} as a potential application area,
  although \citet{sheard01accomp} mentions \emph{mobile code.}
  Analysis of other related work is in section~\ref{sec:related}.} %
\begin{enumerate}
\item Content (text, images, programs, etc.) created
  off-line is uploaded to the server---the \emph{publish}
  stage;
  
\item A user's browser requests content, which is
  transferred from server to client---the \emph{serve}
  stage; and finally
  
\item The content is rendered within the user's browser---the
  \emph{display} stage.\footnote{Increasingly in modern web
    applications, the rendered content is
    \emph{interactive}---responsive to user input without a round-trip
    to the server.}
\end{enumerate}
At \emph{each} stage there is an opportunity for computation to take
place.\footnote{\citet{normark02programmatic} recognized these three
stages as binding times, calling them 
\emph{generated, calculated,} and \emph{dynamic} documents,
respectively.}
Consider the example of a conference calendar, such as the one
illustrated in figure~\ref{fig:confcal}.  After specifying your areas
of interest (perhaps using the ACM classifications), the server
delivers a table of matching conferences, with dates, locations,
deadlines, and links to conference web sites.  Events remain in the
table until a few weeks after they occur, but deadlines that have
passed are marked in red.  You may click on any column header to
change the sort order.  The next time you visit, the server remembers
your preferences.  Perhaps it even sends you email to remind you of
upcoming submission and registration deadlines.

\screenshot{confcal}{A simple conference calendar web service.}
%, implemented as a
%  PHP script that connects to a PostgreSQL database.  It is not as
%  sophisticated as the service described in the text.}

Now, how might this conference service be staged?  Can
anything be computed off-line (at the \emph{publish} stage)?
Yes: since this page is probably part of a much larger site
(whose structure does not change every day), the menus and
other navigation aids can be laid out in advance.  We will
not know which conferences to display until the user
presents some identification (in the form of a `cookie'),
but since the conference data change infrequently, it may
help to prepare the text of each row in advance.

During the \emph{serve} stage, we look up the user's topic
preferences, and ship out just the matching rows.  If we delay
sorting the table until the \emph{display} stage, then the user
ought to be able to adjust the sort criteria without any
further communication with the server.  What about marking
past dates in red?  If this is also delayed until display,
then the code could be cached client-side for long periods
of time, yet still behave dynamically.

At this point, we should emphasize the importance of \emph{profiling}
in developing scalable web services.  This particular design for the
conference calendar may not be optimal, depending on the number of
entries and the relative speeds of the CPU, memory, database, network,
and disk.  Rather, our aim is to provide a single language in which
the various staging possibilities can be expressed naturally.

This approach is in stark contrast to the status quo, where each
system targets one stage only.  The Website Meta
Language\footnote{\url{http://www.thewml.org/}} is an ``off-line HTML
generation toolkit'' designed for the \emph{publish} stage.  But many
other programs (and countless ad-hoc scripts) spit out HTML pages:
\LaTeX2HTML, for example.  Google reports surprisingly many
programs\footnote{\url{http://google.com/search?q=gedcom+generate+html}}
for creating family tree web sites from genealogy database files;
these also count as publish-stage tools.

The \emph{serve} stage is well-served by the ``server page''
languages, including
JSP,
ASP, and
PHP.  The Common Gateway Interface
(CGI)\footnote{\url{http://www.w3.org/CGI/}} addresses the
\emph{serve} stage, as do the embedded interpreters (such as
\texttt{mod\_perl} and \texttt{mod\_python} for Apache) that exist to
ameliorate some of the overhead of CGI.

There is, relatively, a paucity of languages that operate client-side
(\emph{display} stage), probably due to the difficulty of securing an
installed base of interpreters.  JavaScript, Java, and Flash applets are
notable exceptions.  

Imagine implementing the conference calendar, as conceived
above, using currently deployed technology: a Perl script
outputs a PHP page which embeds JavaScript!  Values are
passed from one stage to the next as strings, and the
programmer must manage all the quoting and persistence
issues by hand.

Strictly speaking, these languages are not exclusively confined to the
stages that I have indicated: Javascript can be run server-side, PHP
can be run off-line, and so on.  Nevertheless, migrating code between
stages is hard, and the need for quoting and persistence are
practically show-stoppers.  For comparison purposes (further described
in section~\ref{sec:perf}), we staged some code using PHP.  To achieve
persistence of composite variables from one stage to the next, it
contains gems like this:\label{php-gem}
\begin{verbatim}
    <?= "<? \$list = unserialize(\"".
        addcslashes(serialize($list),'"').
        "\"); ?>\n" ?>
\end{verbatim}
where the \texttt{serialize} library function occurs in stage one and
the \texttt{unserialize} in stage two.  Notice that the \verb+$+ %$
preceding the first occurrence of \verb+list+ is quoted, but the
second occurrence is not.  The \texttt{addcslashes} function is needed
in case the serialized representation contains special characters
(such as double quotes or newlines) that would be misinterpreted by
the PHP parser in the next stage.  Interpreting the stage-one program
guarantees nothing about the well-formedness of the stage-two program
(generated as a string).

We present ``\MOC server pages,'' a new domain-specific language for
web applications programming.  It leverages the staging annotations
and static typing of \MOC~\cite{calcagno03meta,taha00metaml} to
provide safe and precise control over the first two stages.  (We leave
further consideration of the display stage as future work.)  The
system is implemented as two components: a translator transforms the
server page language into a \MOC module, which then can be
incorporated into our multi-threaded HTTP/1.1 server (also written in
\MOC).  The scalability gained by staging certain applications is
stunning: in section~\ref{sec:perf} we describe a directory browsing
service where staging yields a factor of 30 improvement in throughput.
The unstaged version would certainly succumb to the Slashdot effect.

The next section sketches the design and translation of \MOC
server pages, and section~\ref{sec:eg} includes some non-trivial
examples.  The server implementation is decribed in
section~\ref{sec:impl}.  Performance and scalability are
discussed in section~\ref{sec:perf}.

\section{Design}
\label{sec:design}

The general idea of a \emph{server page} language is that we
  write HTML by default, and embed code %
\cd{\Mquote{}like this\Munquote}.
PHP programmers are familiar with this syntax for embedding code, but
in our case, the code itself is written in \MOC.  Here is a trivial
\MOC server page:
\lgrindfile{first.meta}
and its output:
\begin{quote}
  \textit{Hello,} \textbf{Unhandled exception: Failure("Nice try!")}
\end{quote}
The OCaml function \cd{failwith} raises a \cd{Failure} exception
containing the provided message.  If a code block raises an exception,
the message is sent to the client's browser in bold-face, and the rest
of the page is aborted.

In this example and throughout this paper, a {\BGfont sans-serif}
font is used for embedded \MOC code, with {\KWfont bold sans} reserved
for keywords and code delimiters.  A \texttt{typewriter} font is used
for \MOC character strings.  The regular serif font is used for plain
text and HTML within the server page, and for comments within the \MOC
code blocks.

A very common use of code blocks is to print out (i.e., send to the
browser) the result of evaluating some expression.  The syntax
\cd{\Mquote= e\Munquote} is designated for this task; \cd{e} must have
type \cd{string}.  Alternatively, messages may be formatted with
\cd{sprintf} by placing the format string immediately after the code
delimiter.\footnote{In a departure from the standard OCaml syntax for
  \cd{sprintf}, the arguments are comma-delimited.  This way, fewer
  parentheses are required when using the escape or lift operators on
  the arguments.}
%%
\lgrindfile{pi.meta}
The output is:
\begin{quote}
  ........ $\pi$ $\div$ 008 is 0.3927
\end{quote}


One more kind of code delimiter is used for \emph{declarations;}
these are lifted to the top of your program, and evaluated during
the publishing stage:
%%
\lgrindfile{perm.meta}
Output:
\begin{quote}
Permissions on current directory are 0755
\end{quote}

Once published, this output will never change!  The \cd{stat} call is
executed only once (because it is in a declaration block), not on each
request.  This is already a rudimentary kind of staging, but with the
annotations of \MOC, we will gain both flexibility and safety, as
we'll see in the rest of this section.

\subsection{Review of staging annotations}

\MOC augments OCaml with just three annotations, to indicate how
programs are to be staged.  Brackets \cd{\Mbegincode e\Mendcode}
construct future-stage computation.  The code within is not executed
in the current stage of computation, but just returned as a code value
that can be run later.

Within brackets, the splice or escape operator \cd{\Msplice e} may
appear.  It interrupts the code construction to evaluate the expression
\cd{e} (in the current stage) and splice its result into the
future-stage computation.  Thus, \cd{e} is required to evaluate to code of
the proper type.

To compute and splice in a regular (non-code) value, we define a
function \cd{{\KWfont let} lift x = \Mbegincode x\Mendcode}; this
takes any value and turns it into code.  We use it like this:
\cd{\Mbegincode 2 * \Msplice(lift(3+4))\Mendcode} The addition is
performed immediately (because it is escaped), and the result is
spliced into the code, producing \cd{\Mbegincode 2 * 7\Mendcode}

Finally, there is an operator \Mrun{} to execute constructed code.
Applying it to the example above, \cd{\Mrun{} \Mbegincode 2 * 7
  \Mendcode} produces \cd{14}.


\subsection{Translation to \MOC}
To see how all this works, consider how a \MOC server
page is translated into a proper \MOC program, to be executed
at publish time.  Since the program is executed before any
browser has requested the page, it cannot directly return or
output HTML.  Instead, it will construct and return a \cd{code}
object which is subsequently run on each request (serve stage).
The third (display) stage proposed in section~\ref{sec:intro} is not
yet supported by this design.
Figure~\ref{fig:trans.meta} shows a sample \MOC server page,
and figure~\ref{fig:trans.ml} contains its translation.

\mylstf{trans.meta}{This file demonstrates the various
  kinds of code blocks.}

\mylstf{trans.ml}{An automatic translation of the page in
  figure~\ref{fig:trans.meta}.}
%  Declarations have been lifted, and
%  everything else appears within \cd{\Mbegincode brackets\Mendcode}}

Declaration blocks have been lifted to the top; any side effects
contained there are executed when the page is published.  The
constructed code begins on line~5.

The \cd{a b c} represent publish-stage arguments (the names are
specified with \cd{pragma args}), whereas \cd{req} and \cd{puts} are
(fixed) serve-stage arguments.  \cd{req} encapsulates the HTTP request
details, including the headers and query arguments.  \cd{puts} is a
function, provided by the server, to transmit text across the network
to the user's browser.  
%%
The library function \cd{Request.arg} of the type \cd{request\to
  string\to string option} looks up the string value of the request
parameter with a given name.  The \cd{option} type constructor permits
the function to return \cd{None} if there is no matching parameter in
the HTTP request.  On line 6, \cd{arg} is defined as a short-cut for
retrieving a parameter by name.  Since \cd{req} and \cd{puts} are
serve-stage arguments, it is incorrect to use them in the
\emph{publish} (first) stage, and indeed the \MOC type system prevents
this.  These, along with the \cd{lift} function defined near the top,
are essentially primitives from the point of view of the server page
code.%%
%%
\footnote{Here are a few finer points about the translator: it discards
newlines that immediately follow code blocks (otherwise,
figure~\ref{fig:trans.ml} would be dotted with
\cd{puts}~\texttt{"\char"5C{}n"} 
statements).  It automatically appends a semi-colon or the {\KWfont
in} keyword to code blocks, as required (see, for example, lines 7,
11, and 12 in figure~\ref{fig:trans.ml}.)  Finally, because the
translator partially parses the OCaml code blocks, it is not confused
by code delimiters within strings and comments, nor by other uses of
\la{} and \ra{} as operators.}


\subsection{Staged code blocks}
Now that we understand how the server page is assembled into a staged
program, the effects of adding \MOC staging operators to our pages
should be predictable.  Below is another example using \cd{stat}, this
time to display the size of some text file on the server.
With the serve-stage argument \texttt{unit}, we can specify whether
the size should be expressed in bytes (the default), kilobytes, etc.
\lgrindfile{size.meta}%

If this text file does not change frequently (and reporting 
outdated information is no problem), the \cd{stat} and
\cd{float\_of\_int} calls could be lifted into the declaration block,
as with the permissions example.  Furthermore, we may also use
\cd{lift} and the splice operator to perform the divisions in advance,
even though they are underneath the {\KWfont match} (which cannot
happen until the serve stage): \lgrindfile{size2.meta}%

Still, the \cd{printf} conversion is performed at serve time, so maybe
it is best to pre-generate that as well:%
\lgrindfile{size3.meta}%

Now, all that remains to execute at serve time is to check the dynamic
\cd{unit} argument and spit out one of three pre-generated strings.
The generated code block looks something like this (substantially
cleaned up from the \MOC pretty-printer, with references to persistent
values resolved in-place):
\lgrindfile{size-out.ml}%

To handle more complex situations, code blocks may be constructed
conditionally and recursively.  Here is an example that
prints a count-down, but removes the loop overhead by expanding to a
sequence of 99 \cd{puts} statements.
\lgrindfile{count.meta}%

The brackets around the \cd{puts} on the last line are necessary,
because \cd{count} splices the \cd{puts} call into the code it
generates.  Omitting the brackets would result in a compile-time type
error.

These tiny examples suggest the ease with which the `boundary' between
the stages can be adjusted, just by tweaking the staging annotations.
Moreover, the static typing of \MOC ensures in advance that our
programs generate type-correct code only.  
%%
The examples also illustrate some of the common uses of the escape
operator, for which we developed syntactic sugar.  The first set of
code blocks we define use the tilde character to indicate that some
part of the enclosed code is escaped:

\begin{tabular}{l@{\quad$\leadsto$\quad}l}
\cd{\Mquote{\Mtilde} a \Munquote} &
\cd{\Mquote{} \Msplice( a ) \Munquote}
\\
\cd{\Mquote{\Mtilde=} b \Munquote} &
\cd{\Mquote{=} \Msplice( b ) \Munquote}
\\
\cd{\Mquote{\Mtilde}{\KWfont let} x = c \Munquote} &
\cd{\Mquote{} {\KWfont let} x = \Msplice( c ) \Munquote}
\\
\cd{\Mquote{\Mtilde}\id{"fmt"} d, e, f\Munquote} &
\cd{\Mquote{}\id{"fmt"} \Msplice(d), \Msplice(e), \Msplice(f)\Munquote}
\\
\end{tabular}

where the \cd{a} has type \cd{unit code}; \cd{b} has type
\cd{string code}; and the types of \cd{d,e,f} match the
format string.

It is also common to use the escape with \cd{lift}; this means we are
computing a value immediately and splicing it into the code.  For
these blocks, we use the \Mbang{} character:

\begin{tabular}{{l@{\quad$\leadsto$\quad}l}}
\cd{\Mquote{\Mbang} a \Munquote} &
\cd{\Mquote{} \Msplice(lift( a )) \Munquote}
\\
\cd{\Mquote{\Mbang=} b \Munquote} &
\cd{\Mquote{=} \Msplice(lift( b )) \Munquote}
\\
\cd{\Mquote{\Mbang}{\KWfont let} x = c \Munquote} &
\cd{\Mquote{} {\KWfont let} x = \Msplice(lift( c )) \Munquote}
\\
\cd{\Mquote{\Mbang}\id{"fmt"} d, e\Munquote} &
\cd{\Mquote{}\id{"fmt"} \Msplice(lift(d)), \Msplice(lift(e))\Munquote}
\\
\end{tabular}

In these cases, the expressions should not have code types: \cd{a}
simply has type \cd{unit}, \cd{b} has type string, etc.  This is the
only difference between the \cd{\Mtilde} and \cd{!} variants; both are
evaluated in the first stage.  We will see examples of most of these
blocks in section~\ref{sec:eg} and appendix~\ref{sec:dir.meta}.

\subsection{Publish-stage arguments}

\MOC server pages support publish-stage arguments, as demonstrated by
the identifiers \cd{a b c} in figures~\ref{fig:trans.meta}
and~\ref{fig:trans.ml}.  The programmer specifies the pattern to be
used with a \cd{\Mquote{}pragma args ...\Munquote} declaration, which
may appear anywhere in the code.  Any identifiers following \cd{args}
in the \cd{pragma} declaration will become publish-stage parameters.

With publish-stage parameters, the page may be instantiated in
countless ways.  Continuing the count-down example, we could map the
URI \id{/longcount} to a code block generated with an argument of 99,
while \id{/shortcount} refers to code with the argument 9.  Both are
generated from the same server page.

\lgrindfile{count2.meta}

Mapping from URIs to instantiated \MOC code is, for now, left as an
implementation detail (see section~\ref{sec:impl}).

\section{Examples}
\label{sec:eg}

To explore the expressiveness of this design, we now look at a series
of web services, organized to comprise a small web site.  To give
the services a similar look, we developed a site-wide style sheet, and
an OCaml function to generate a navigation bar from a hierarchical
list of page titles and links.  
%%
\mylst{navspec-sig.ml}{These functions help define the standard site
  layout.}
%%
The page layout functions in figure~\ref{fig:navspec-sig.ml} may be
invoked at any stage.  Naturally, if the page title includes a dynamic
argument, then \cd{preamble} will have to be delayed until the serve
stage.

\subsection{The ubiquitous power function}
\label{sec:eg:power}

\screenshot{power127}{Web browser displaying the result of
  7\textsuperscript{127}.}
%  The URI \texttt{/power127} maps to code
%  where the early argument (the exponent) has been set to 127, and the
%  late argument (\texttt{x}, the base) is taken from the query
%  string.}

Judging from its prevalence in the multi-stage programming literature,
we are certain that millions of grateful users would subscribe to an
online service capable of computing the \emph{power} function.  The
screen shot in figure~\ref{fig:power127} illustrates how it works.
The navigation bar shows the exponents for which code has been
pre-generated.\footnote{In the current implementation, it is not
  possible for the user to request other exponents once the server is
  running.  See section~\ref{sec:impl} for an explanation.}  After
selecting the exponent, the user types the base into the form and
presses return.  The result is computed using the arbitrary-precision
\cd{Num} module of the OCaml library.  The complete script appears in
figure~\ref{fig:power.meta}.

\mylstf{power.meta}{The staged power function
 as illustrated in figure~\ref{fig:power127}.}
%  web service.  The exponent \cd{y} is an early (publish stage)
%  argument; but the base \cd{x} is late (serve stage).}

The user's input is converted to a number relatively late in the
script (line 21).  If \cd{num\_of\_string}
generates an exception (perhaps because the user typed non-numeric
text into the box), the navigation bar and form will have already been
output before the error message appears.

In section~\ref{sec:perf}, we will measure the impact of staging on
the scalability of this application.  To derive the un-staged version
for comparison, we simply replace \cd{\Mquote{\Mtilde}} and
\cd{\Mquote{!}} blocks with plain \cd{\Mquote{}} and remove all other
brackets and escapes from the code in figure~\ref{fig:power.meta}.

\subsection{Directory browsing}
\label{sec:eg:browse}

Now we consider a more substantial example.
Many web servers can be configured to permit clients to browse
directories.  The web server generates, on the fly, an HTML page
containing the names and attributes of (and hyperlinks to) all
the files in whatever directory is specified by the URL.  In Apache, the
\texttt{mod\_autoindex} module provides this feature.
%%
ViewCVS is a more
complex example of the same idea; it allows remote users to browse a
CVS repository with a standard web browser.

This kind of service can be fairly resource intensive; each HTTP
request is likely to generate dozens of system calls and disk
accesses.  If the directory is viewed more often than it is changed,
then it makes sense to cache or pre-generate the pages.  Although it
is simple enough to write a script to generate static directory pages
off-line, what if we also want dynamic behavior, such as
user-controlled sorting and filtering?

Our implementation not only lists files in a given directory, but
displays their MD5 checksums, renders their sizes in human-readable
form (`1.2M' rather than `1194822'), and colors their names based on
their extension or file type.  We also support dynamic (serve time)
customization: the user may specify a regular expression for
filtering, and select one of 5 criteria for sorting.  The screen shot
in figure~\ref{fig:server-dir} shows the result of browsing the source
directory of the server itself.

\screenshot{server-dir}{Browsing the server source code directory.}
%  File names are colored according to their \emph{kind,} which is
%  computed from the extension and the file type.  See also
%  figure~\ref{fig:dir.meta}.}

This service is a bit like the conference calendar proposed in
section~\ref{sec:intro}.  We pre-generate \emph{everything} that does
not rely on serve-time parameters.  The stats and MD5 sums of the
files are collected \emph{once,} in the publish stage.  The text of
each possible row is prepared in advance.  Once the user's request is
made, we filter and sort the list, then output the pre-generated text
of each remaining row.  The files need not be opened or even
\id{stat(2)}ed while the user is waiting.  Appendix~\ref{sec:dir.meta}
contains the complete script, with documentation.

Much of the code is unaffected by staging; helper functions (that sort
according to the selected criterion, format the human-readable sizes,
and collect the file information) are oblivious to the stage in which
they are run.  Therefore, the staged code comprises a relatively small
portion of the entire program: mainly the function \cd{list\_files}
and its call site.

To understand the staging technique, it may help to examine the output
of just the first stage of computation.  The code block in
figure~\ref{fig:dir-out.ml} is cleaned up from the \MOC
pretty-printer, with references to compiled values substituted
in-place.  
%%
\mylst{dir-out.ml}{Generated code for directory browser; see also
  appendix~\ref{sec:dir.meta}.} 
%%
The \cd{list} is formed by testing filenames (in reverse alphabetical
order) against the compiled regular expression \cd{rc}, prepending only
those that match.  The file info record for each file has already
been prepared and appears in the code as a value.

Because the file information is compiled into the code, any changes to
the file system following the first stage of execution will \emph{not}
automatically appear on the web interface.  To see the latest files,
we need to re-run stage one.  This cannot be done from within the \MOC
server page itself, but we have programmed the server to regenerate
pages whenever the `\cd{!}' character is appended to the URI.  The
directory browser pages feature a `Regenerate' button at the bottom
which will bring them up to date by running the publish stage code
again, and caching the result for subsequent requests.



\subsection{Server introspection}

Some web servers can be configured to display their status in response
to certain URIs (such as \id{/server-status} on Apache).  We
programmed a few status services in \MOC.  The screen shot in
figure~\ref{fig:gc} displays garbage collection statistics from the
OCaml \id{GC} module.  The only thing pre-generated here is the
navigation bar.

\screenshot{gc}{Garbage collector statistics, from the OCaml \id{gc}
  module.}

%\lgrindfile{gc.meta}

\section{Implementation}
\label{sec:impl}

The implementation consists of two parts: a translator and a web
server.  The translator transforms the server page syntax into plain
\MOC, as illustrated in figure~\ref{fig:trans.ml}.  It recognizes all
the server page blocks defined in section~\ref{sec:design} but does
not completely parse the \MOC code contained within them.  Instead, it
recognizes just enough of the keywords and delimiters to decide
whether to add semi-colons or `{\KWfont in}' after each block.  This
approach makes the translator fairly robust to minor changes in OCaml
syntax.

The disadvantage of this method is that very few syntax errors (and
no type errors) are detected by the translator itself.  So, errors
reported by \MOC are displayed in terms of the translated program, not
the source program.

The server component is much more complex.  We implemented the
essential parts of the HTTP/1.1 specification as a multi-threaded
OCaml program.  Upon receiving a request for some URI, the server uses
a \emph{chain of responsibility}~\cite{gamma95design} to determine
how to handle it.  The chain is specified as a parameter to the
server.  Two primary handlers are provided: \cd{FileHandler} and
\cd{CodeHandler}.

The \cd{FileHandler} takes a file system path as a parameter, and then
interprets each URI as a file name relative to that path.  If such a
file exists, it sends it verbatim to the client.  If not, it passes
the responsibility on to the next handler in the chain.

The \cd{CodeHandler} is provided with a dictionary to map URIs to code
values.  It looks up the URI in the dictionary, and if a match is
found it invokes that code.  For now, the map is hard-coded at build
time and the code values are already in memory when the server starts
accepting requests.  This is because \MOC does not currently permit
programs to read and write code values to a file.

Here are the types of the page code and the map, along with the
signature for the \cd{CodeHandler}:

\lgrindfile{codeHndl.mli}
%%
\cd{page} is the type of the code that is constructed by each server
page; recall that the page takes two serve-time parameters: one
encapsulating the HTTP request, the other is the \cd{puts} function
for sending text back to the client.  

Data structures of type \cd{map} tell the server which pages are
mapped to which URIs.  The range of the map is a pair: the first
component (of type \cd{unit\to page}) is a function that will re-run
stage one computation (using the \Mrun{} operator internally); the
second component (of type \cd{page ref}) is a cell where the latest
page is cached.

The \cd{unit\to page} formulation is a work-around for what might
otherwise be expressed as \cd{page code}, and run from within the
\cd{CodeHandler}.  Unfortunately, code values in MetaOCaml must remain
polymorphic to run them; the type is actually \cd{('a, page) code},
for all \cd{'a}.  This is traditionally problematic in ML: we can
define a polymorphic data structure, but not a data structure
containing polymorphic values.  The work-around we used captures the
polymorphic code value in a closure, which is then added to the map.
Another possibility is to use the limited form of explicit
polymorphism supported in OCaml, with the syntax \cd{\{f : 'a. ('a,
page) code\}}.\footnote{Thanks to an anonymous reviewer for pointing
this out.}  To use code values more directly, we would need rank-2
polymorphism; work by \citet{garrigue99semi} is headed in that
direction.


\section{Performance}
\label{sec:perf}
In this section, we describe the performance
characteristics of the prototype, focusing in particular on the
benefits of staging various web services.
%%
The single most important metric for evaluating web server performance
is \textit{throughput:} the number of requests successfully answered
per unit of time.  To establish a baseline, we first tested the
throughput of our custom OCaml HTTP server delivering chunks of
static data of various sizes, up to 64k bytes.
%%
Measurements were taken on an otherwise idle 1.8GHz Intel Xeon
workstation\footnote{with 512kB cache, 768MB RAM, and Ultra160 SCSI}
running Linux 2.6.  We used \texttt{ab}, the Apache HTTP server
benchmarking tool, to issue requests from 8 threads simultaneously for
30 seconds.\footnote{invoked like this: \texttt{ab -k -t 30 -c 8
  }\textit{url}}

\gnuplot{static}{Throughput for static pages.  Note the logarithmic
  scale on the \textit x axis.}

The baseline results are shown in figure~\ref{fig:static}.  In the
``OCaml FileHandler'' series, the files were treated just as static
files, opened on disk, and copied out to the socket.  There are two
sets of results for the FileHandler: one compiled as native code, and
the other compiled as byte code.  A native code compiler for MetaOCaml
is not yet available.  Therefore, most comparisons in this section
will rely on byte code.  The native code results we are able to obtain
at this time suggest how much improvement we can expect once native
code is an option.

For the ``\MOC CodeHandler'' series, the same set of files were
treated as \MOC Server Pages, and thus translated (in advance) into
one big \texttt{puts} statement, to be executed (as byte code) by the
\texttt{CodeHandler} module.  The only reason the code beats the (byte
code) FileHandler in the beginning is that all code pages are already
loaded into the server's memory on startup (to work around a
limitation of the prototype---see section~\ref{sec:impl}), but the
FileHandler must read files from disk each time.

Figure~\ref{fig:static} includes comparable results for PHP, the
popular server-side computation system.\footnote{libphp4.so
  (version 4.3.10) loaded into Apache 1.3} Here, we gave the same
static data files the extension \texttt{.php}, so that Apache
would treat them as PHP code, even though they have no
\cd{\Mquote{}code blocks\Munquote}.
%%
We omitted the results for Apache serving static files, because they
are way off scale: Apache handled an astounding 4,438 hits per second
for the 1k file, and 1,813 for the 64k file.  Apache is heavily
optimized for serving static files: apart from caching, it uses the
special \texttt{sendfile(2)} system call for zero-copy file transfer
from kernel space~\cite{tranter03sendfile}.  Although our prototype is
no match for Apache on static files, the overhead for interpreting
code seems no worse than that of PHP.
%%

Using the same methodology, we now consider the performance of the
staged and unstaged power functions; see figure~\ref{fig:power}.
Since we are comparing staged vs. unstaged (and not native vs. byte
code), these results need to be interpreted in two distinct groups.
The top two lines are native code, while the bottom three lines are
byte code.  Start with the byte code.

\gnuplot{power}{Staged versus unstaged power function.  The
  server computes $2^x$.}

The throughput for the pages staged with MetaOCaml is about 30\%
higher than the unstaged version in the beginning, but as the
exponents increase (again, note the log scale in the graph) the gap
narrows.  In this program, staging removes the loop overhead, but the
cost of the multiplications and conversion to a decimal string (which
involves non-trivial divisions) are needed either way.  Eventually,
the cost of those operations dominates everything else.\footnote{The
final result,
  2\textsuperscript{8191} has 2,466 digits in base 10.}

\mylst{powerhof.ml}{Staging the power function using higher-order
functions.} 

At the suggestion of an anonymous reviewer, we also tried staging
using higher-order functions in OCaml instead of the code splicing
features of \MOC.  The relevant fragment of code is shown in
figure~\ref{fig:powerhof.ml}.  It
recursively builds up a closure that represents the computation
required to compute any number to the power \cd n.

As one might expect, staging using higher-order functions is slightly
worse than producing specialized code as in \MOC, but better, of
course, than no staging at all.  Moreover, this version compiles
easily to native code with the standard OCaml tools, and this improves
its performance significantly.  We do have reason to hope, however,
that the native \MOC version, once working, will ultimately perform
the best.

Finally, we look at the performance of staged and unstaged directory
browsing; see figures~\ref{fig:browse} and~\ref{fig:hofbrowse}.  Here,
we created directories containing fixed numbers of files with random
data.  The average file size was 32k.  The \emph{x}~axis shows the
number of files in each directory.

\gnuplot{browse}{Staged versus unstaged directory browsing, in \MOC
  and PHP.}

\gnuplot{hofbrowse}{Directory browsing staged with higher-order
functions vs. \MOC.}

Besides the staging factor, two implementation languages are compared:
we implemented the same functionality in PHP.  The `staged PHP' version must
be run first from the command line; this outputs a PHP script
which is then run by the server.  With PHP we must, as noted in
section~\ref{sec:intro}, manage the quoting and persistence 
by hand.  (This program is the source of the horrible
\texttt{serialize}/\texttt{unserialize} code shown on
page~\pageref{php-gem}.)

Directory browsing was the most realistic of the examples, and the
benefit of staging is crystal clear.  In browsing a directory of 64
files, the unstaged programs barely answered 18 requests per second.
They would certainly succumb to the Slashdot effect---and compilation
to native code made essentially no difference.

This is exactly the kind of page where the real work needs to be done
in advance.  But that does not mean it needs to be a completely static
page, either.  By carefully staging the computation, we gathered the
file information in advance, yet still filtered and sorted the results
on demand.  The staged \MOC directory browser answered more than 550
requests (for the same 64-file directory) per second.

The degradation in performance of the staged PHP version is most
likely due to the fact that the stage two script must be \emph{parsed}
on each request; the size of the script is proportional to the number
of files in the directory.  With \MOC, the code size is also
proportional to the number of files, but the values are byte-compiled
(and already in memory).  At any rate, our goal is not to beat PHP on
performance, but rather to gain the performance benefit of staging
\emph{without} the awkwardness of staging in a language (like PHP) that
does not support it.

We also tried staging the directory service using higher-order
functions, the same technique as described for the power function.
Again, this did not quite match the performance of the \MOC version
(when comparing byte-code to byte-code).  However, in
figure~\ref{fig:hofbrowse}, we can also see the substantial difference
that compilation to native code can make.  We expect that the native
\MOC compiler will enable even better throughput.


\section{Related and future work}
\label{sec:related}

Most of the \emph{server page} systems embed programs within web pages
using similar techniques; examples include JSP, ASP, PHP,
MSP~\cite{elsman03web} (based on SML), and AS/XCaml (based on
OCaml).\footnote{Application System Xcaml, by Alessandro Baretta.
  \url{http://www.asxcaml.org/}}
%%
Ours appears to be the first such language with
explicit support for \emph{staging} the computation.  We inherited the
staging annotations of \MOC~\cite{calcagno03meta} and designed several
new kinds of code blocks based on them.

There is much related work on using various features of modern
programming languages to implement sophisticated web services.
\citet{normark02programmatic} proposed writing web pages in Scheme
using the Lisp Abstracted Markup Language (LAML), which essentially
represents HTML documents as S-expressions.  He distinguished three
different binding times, when the Scheme program could be evaluated:
off-line, page access time, or browse time (client-side).  These
correspond precisely to the three stages we identified in section 1,
but his programs did not \emph{transcend} different stages.  It is
possible to implement staged programming in Scheme using
\cd{quasiquote}, \cd{unquote}, and \cd{eval}~\cite{davies01modal};
this would lead to a similar capability, save for the difference
between static and dynamic checking of the generated code.

\citet{takebe02caching} developed a partial evaluation technique for
PHP.  Their tool, PHP-Mix, reads standard PHP source code and performs
binding time analysis.  Then, it generates a semi-static script as
output.  It understands a significant subset of the PHP function
library, including the database connectivity.  Unfortunately, some
properties of PHP---such as the lack of variable declarations and the
single global scope---cause the analysis to be overly conservative,
so it does not always achieve the optimizations we desire.

\citet{queinnec00influence} and \citet{hughes00generalising} observed
that multi-page web services could be implemented more naturally using
continuations and \cd{call/cc} (essentially, by treating the server
and user as coroutines).  This way, we can treat the entire service as
one program---that suspends itself while waiting for user
input---instead of developing each page as a separate program.  This
is a valuable technique, and appears to be orthogonal to staging;
although it would be worth implementing both together to determine if
there are unforeseen interactions.

\citet{graunke01programming,graunke03modeling} took up this idea and
mixed it with other language features: first-class modules, preemptive
threads, and \emph{custodians,} for managing resource consumption.
The result is a server that achieves lower overhead for dynamic
services compared to Apache CGI (although the overhead of CGI is
avoided by most modern dynamic services, by embedding interpreters for
JSP, PHP, or similar in the server).  \citet{matthews04automatically}
compile such direct-style interactive programs into CGI-style scripts.


Unlike more ad-hoc staging methods, \MOC server pages guarantee the
type safety for generated code up front.  We do not, however, make any
guarantees about the validity of the generated HTML.
\citet{elsman04typing}, \citet{wallace99haxml}, \citet{hosoya03xduce},
and Ohl\footnote{XHTML module.
  \url{http://theorie.physik.uni-wuerzburg.de/~ohl/xhtml/}} 
%%
leverage ML-like type systems to validate (X)HTML generators.
Integrating their ideas into \MOC server pages may permit validation
of the generated document as well.


Our server page language is defined by translation into \MOC.
Unfortunately, this means that error messages refer to the translated
code, not the original, embedded code.  Camlp4 is a flexible
pre-processor capable of modifying the concrete syntax of OCaml while
maintaining usable error messages.\footnote{Camlp4, by Daniel de
  Rauglaudre.  \url{http://pauillac.inria.fr/caml/camlp4/}}
Formulating the server page syntax with Camlp4 would likely be an
improvement over the current, ad-hoc translator.

In motivating \MOC server pages, we observed that the computation
associated with a web page naturally decomposed into three stages:
publish, serve, and display.  Our language, however, was designed to
support just the first two stages.  It is straightforward to imagine
an extension to the third stage (since \MOC itself has no constraints
on the number of stages) but the implementation may be tricky.  First,
we must overcome the process boundary between server and client.  We
currently have no way to export \MOC code blocks from the program that
created them into another program.  (On the server, we circumvented
this limitation by running the publish and serve stages within one
process.)  Next, we need a browser capable of loading and executing
\MOC code.  \citet{rouaix96web} demonstrated a browser called MMM,
written in Caml, that could run applets loaded as Caml byte-code.  It
may be possible to update his browser for use with \MOC, but
developing a plug-in that works with conventional browsers would be
preferable.

One of the major shortcomings of our implementation is that all the
pages must be loaded into memory when the server starts.  Ideally, a
separate tool would translate the server page source directly to
byte-code, stored in an image on disk.  Then the server would map URIs
to these byte-code files, loading and executing them on demand.
Whenever we want to re-run the publish stage, we could do so
independently of the server.  

Another interesting avenue is to incorporate work on
\emph{offshoring}~\cite{eckhardt04offshoring}.  An alternative
\emph{run} construct translates generated code blocks to lower-level
languages for improved performance.  In the domain of web services, it
is conceivable that a \MOC server page could then generate (via
offshoring) a specialized server page in a more mainstream language,
such as PHP, JSP, or---for client-side computation---JavaScript.

More substantial applications are needed to demonstrate both the
performance gains and the expressiveness of this approach.  A web
service that uses a database and is spread over several pages would be
more realistic.  \citet{queinnec00influence} and
\citet{graunke01programming} use features of functional languages to
express user interactions more naturally, given the stateless nature
of HTTP.  We believe that these techniques are orthogonal to the
staging features provided by \MOC.

Finally, although our HTTP implementation stands up to the Apache
benchmarking tool, it is not likely to win any awards for reliability
or flexibility.  Our ideas would likely have greater impact if they
were implemented as a module for a real server such as Apache or
AOLserver.  Beyond the concerns outlined earlier in this section, this
is likely to be `just' an engineering effort.


\section{Conclusion}
\label{sec:concl}

Web publishing is an important application domain that is naturally
staged.  Web programmers write staged programs, but they do it the
old-fashioned way: one script outputs another as a string.  This is a
tremendous opportunity for multi-stage languages.

We presented the design of \MOC server pages, a new
domain-specific language for web applications programming.  It
leverages the staging annotations of \MOC to provide safe and
precise control over the each stage of the computation.
%%
We have shown the substantial benefits of this approach in terms
of performance and expressiveness, although the prototype
implementation suffers some limitations because it is unable to
read and write generated code to a file.

\section*{Acknowledgments}

Sincere thanks to the anonymous reviewers, for many helpful and
enlightening recommendations on both the work and the presentation.
%%
Thanks to my M.S. student, Kiichi Takeuchi, for exploring some
applications of these ideas, and for helping me understand the work of
\citet{takebe02caching} (written in Japanese).
%%
Thanks to Walid Taha for organizing the \MOC workshop, and to the rest
of the \MOC team for their development efforts.

\bibliographystyle{abbrvnat}
\bibliography{refs,../../bib/library}
\appendix
%\twocolumn
\section{The directory browser}
\label{sec:dir.meta}
%\raggedright
\setcodefonts{11pt}{10pt}
\LGnuminterval=5000
\LGindent=8pt
\gdef\lgcommentstart{\end{lgrind}\vskip-3pt}
\def\lgcommentend{\vskip.5\baselineskip\begin{lgrind}}

\begin{lgrind}\input dir.meta
\relax\end{lgrind}

\end{document}

%%% Local Variables: 
%%% mode: latex
%%% TeX-master: t
%%% End: 
